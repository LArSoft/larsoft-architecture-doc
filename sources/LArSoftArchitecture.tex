\documentclass{article}


\usepackage{graphicx}

\usepackage{hyperref}

% for internal referencing
\usepackage{cleveref}

% to comment out entire blocks
\usepackage{comment}

\usepackage{xspace}


%-------------------------------------------------------------------------------
%--- the usual endless set of definitions

\newcommand{\ART}{\textsl{art}\xspace}
\newcommand{\NOvA}{NO$\nu$A\xspace}

%-------------------------------------------------------------------------------


\begin{document}

\title{The LArSoft architecture}
\author{The LArSoft project}
\maketitle

\tableofcontents

\section{Introduction}
\label{sec:introduction}

% \section{Introduction}
% \label{sec:introduction}

\subsection{Purpose}
\label{ssec:purpose}

The LArSoft toolkit is designed to enable simulation, reconstruction and
physics analysis of data from any detection system based on Liquid Argon
TPCs. Its common tools and algorithms render the development and
analysis process more uniform across the Experiments. LArSoft is
extensible to accommodate evolving Experiments' needs and adoption by
new Experiments.

This document describes the current architecture of LArSoft toolkit.
The architecture was developed according to, and therefore reflects,
the consensus of LArSoft partners, including the adopting Experiments.

The document provides a reference for the reader interested in learning
the general structure of LArSoft, its functional areas and interactions
with the execution environment.
It also offers guidelines for the contributor aiming to develop new algorithms
within LArSoft or to use it together with external tools.


\subsection{Scope}
\label{ssec:scope}

This document provides an overview of the architecture of LArSoft
toolkit, including its relationship with the surrounding software
environment. The internal flow of the different subsystems is also
described. The document intends to capture and convey the significant
architectural decisions which have been made on the system, that reflect
into the current implementation or drive its development.\\

This document describes the architecture of LArSoft to date. It includes
description of the communication protocols with libraries it relies on
and with packages LArSoft cooperates with. The design and flow of
different components is described.

Some commonly used LArSoft elements are employed to exemplify
flows and connections, but this is no attempt to exhaustively describe
each, or any, of the single elements.




\section{Overview}
\label{sec:Overview}

% \section{Overview}
% \label{sec:Overview}

% 
% this is my reinterpretation of the "use case view" suggested by Saba
% 

The LArSoft toolkit aims to offer a solution for the typical data analysis scenarios
of an experiment based on a Liquid Argon TPC detector:
\begin{itemize}
  \item generation of physics events
  \item simulation of physics processes in the detectors
  \item simulation of detector readout response
  \item reconstruction of low and high level physics objects
  \item analysis and presentation of collected data
  \item graphical display of physics events
\end{itemize}

As an example, suppose a scientist may want to develop a new clustering algorithm
optimized for a certain type of physics events.
LArSoft offers interface to generators to produce either simplified physics events
or more realistic ones that include, for example, cosmic radiation.
It also provides the simulation of those events in the specific experiment detector.
If the target processes are common enough, the experiment might have already
executed these steps on large scale, also using LArSoft, and provided the necessary input.
The scientist is then presented with standard interfaces to access geometry and detector
information, and standard data structures to start from,
including hits on single wires suitable as starting point for a clustering algorithm,
and to store the results into.
She (or he) can use the standard framework environment to write an algorithm class
and its framework module, compile it and test it immediately on simulated data.
The clusters, saved in a standard LArSoft data structure,
can be immediately visualized in a event display,
and adjust the code as needed for the next development iteration.
Depending on the algorithm, the time between a code change and the
visualization of its effect may take less than one minute.
Finally, replacing the input with actual detector data, that uses the same format
as the simulation, she will immediately see the performance in the real case.

As a different example, a scientist may want to compare two different algorithms
analyzing reconstructed tracks.
After the tracks are produced, running the track reconstruction algorithm
on either simulated or real data, she will write one or more analysis algorithms
and their framework modules to produce the necessary plots.

% Since LArSoft has been designed to take advantage of the \ART framework\cite{ART},
% LArSoft users will extensively work with its concepts,
% including for example services and modules, and infrastructure, like \ART scripts
% to create skeleton modules, and the configuration based on FHiCL language\cite{FHiCL}.



\section{Logical view: components}
\label{sec:Components}

% \section{Logical view: components}
% \label{sec:Components}

%
% This is the merging of "component view" and "component interafctions"
% from J&M outline, following Saba's suggestion.
%
%

To provide the best solutions for LAr TPC simulation, reconstruction and
analysis of data, LArSoft interacts with other software aimed to
provide developers with tools commonly in use by the broader physics community,
standardize code development,
and allow for experiment-specific needs (\cref{fig:LArSoftRelations}).
\begin{figure}
	\centering
	\includegraphics[width=0.8\textwidth]{figures/LArSoftEnvironment}
	\caption{\label{fig:LArSoftRelations}
		Relationship of LArSoft with other software categories.
	}
\end{figure}

Physics developers typically rely on copious libraries providing general or physics-specific services
(\cref{fig:LArSoftRelations:Libraries}).
LArSoft already offers:
\begin{itemize}
	\item access to a framework, \ART~\cite{ART}, providing essential functionalities
		including an event data model, an event loop, workflow definition and control,
		plug-in of code, distribution and tracking of job configuration,
		serialization of the results, and more
	\item proxy-, web-based access to data bases via \libwda~\cite{libwda},
		or direct access to \PostgreSQL databases\footnote{%
		Experience has shown that direct database does not scale well with the number of accessing jobs.%
		}
	\item physics libraries, as CERN \CLHEP and \nutools~\cite{nutools}
	\item event generation packages: \GENIE~\cite{GENIE}, \CRY~\cite{CRY}, HEPEVT files
	\item GEANT4~\cite{GEANT}, the detector simulation library
	\item pattern recognition libraries, like \Pandora
	\item data analysis tools, like CERN \ROOT~\cite{ROOT}
	\item visualization aids, also with CERN \ROOT and \nutools
\end{itemize}
\begin{figure}
	% TODO align the two sides;
	% 
	% \newsavebox{\tempbox}
	% \sbox{\tempbox{\includegraphics{large figure}}}
	% \subfloat{\label{...}\usebox{\tempbox}}
	% \subfloat{\label{...}\vbox to \ht\tempbox{\vfil\includegraphics{small figure}\vfil}}
	% 
	\subfloat{
		\label{fig:LArSoftRelations:Libraries}
		\includegraphics[scale=0.04,trim=500 500 500 500]{figures/LArSoftAndLibraries}
	}
	\subfloat{
		\label{fig:LArSoftRelations:Experiments}
		\includegraphics[scale=0.04,trim=500 500 500 500]{figures/LArSoftAndExperiments}
	}
	\caption{\label{fig:LArSoftRelationsI}
		Relationship between LArSoft and \protect\subref{fig:LArSoftRelations:Libraries} third-party libraries and \protect\subref{fig:LArSoftRelations:Experiments} experiment-specific software.
	}
\end{figure}

LArSoft also needs to accommodate specific needs from the experiments.
Experiments directly contribute LArSoft content when it's suitable, that is,
of general utility and experiment-agnostic.
In the other cases, experiments interface to LArSoft though many channels
(\cref{fig:LArSoftRelations:Experiments}):
\begin{itemize}
	\item detector geometry is provided in GDML or ROOT format
	\item detector conditions can be learned via static configuration or from experiment databases
	\item detector data is acquired by special \ART modules or by standard \ART files (e.g., produced by \ARTDAQ) containing standard LArSoft data products
	\item specialized services and algorithms can be plugged in using the \ART framework
	\item job configuration, controlling the data to be processed and the sequence of actions to perform,
		is specified in \FHiCL language
	\item workflows are defined by the experiment, typically by using custom scripts that include the execution of LArSoft main program
\end{itemize}
\begin{figure}
	\centering\includegraphics[scale=0.04]{figures/LArSoftAndDevelopment}
	\caption{\label{fig:LArSoftRelations:Development}
		Relationship between LArSoft and development software categories.
	}
\end{figure}

LArSoft has a large number of interdependent components,
and provides the users tools to facilitate code development (\cref{fig:LArSoftRelations:Development}).
LArSoft code is organized in repositories that can be compiled when needed.
A building system ensuring builds consistent among all supported platforms is employed.
The \UPS~\cite{UPS} distribution system ensures that the same consistency is preserved at run time.
Infrastructure for automatic execution of user tests is also provided,
together with a growing number of tests exercising parts of LArSoft tools.



\subsection{Internal components}
\label{ssec:Components}

Most LArSoft components can be grouped into some broad functional categories.
Some of them are well established, while others are being developed now or have been just designed.
The following list touches the main ones, without being exhaustive:
% \begin{itemize}
% 	\item physical constants
% 	\item helpers for common framework usage patterns (\eg, creation of asociations between data products)
% 	\item detector geometry description
% 	\item persistent data structures (``data products'')
% 	\item detector information services: liquid argon and detector properties, detector clocks, readout channel quality
% 	\item calibration services: readout channel pedestals
% 	\item physics event generation
% 	\item detector simulation: TPC and optical detectors
% 	\item readout simulation: template modules for TPC and optical detectors
% 	\item simulation of optical triggers
% 	\item calibration: template modules, description of residual electric charge in TPC volume
% 	\item object reconstruction: 1D (TPC wire hits, optical hits), 2D (TPC hit clusters), 3D (tracks, showers, vertices) and time (optical flashes)
% 	\item TPC hit simulation and correlation between reconstructed objects and generated particles
% 	\item energy and momentum reconstruction (``calorimetry'')
% 	\item particle identification
% 	\item global event reconstruction
% 	\item graphical display of generated and reconstructed objects (``event display'')
% 	\item analyser module example
% \end{itemize}
\begin{description}
	\item[detector information] \mbox{} % make it start a new line
		\begin{itemize}
			\item detector geometry description
			\item detector information services: liquid argon and detector properties, detector clocks, readout channel quality
			\item calibration services: readout channel pedestals
			\item map of residual electric charge in TPC volume
		\end{itemize}
	\item[persistent data structures] (``data products'')
	\item[operations] \mbox{} % make it start a new line
		\begin{itemize}
			\item physics event generation
			\item detector simulation: TPC and optical detectors
			\item readout simulation: template modules for TPC and optical detectors
			\item simulation of optical triggers
			\item calibration template modules
			\item object reconstruction: 1D (TPC wire hits, optical hits), 2D (TPC hit clusters), 3D (tracks, showers, vertices) and time (optical flashes)
			\item TPC hit simulation and correlation between reconstructed objects and generated particles
			\item energy and momentum reconstruction (``calorimetry'')
			\item particle identification
			\item global event reconstruction
		\end{itemize}
	\item[programming utilities] and framework interface
		\begin{itemize}
			\item physical constants
			\item helpers for common framework usage patterns (\eg, creation of asociations between data products)
		\end{itemize}
	\item[graphical display] of generated and reconstructed objects (``event display'')
	\item[example] of analyser module
\end{description}



\section{Process view: workflows}
\label{sec:Workflows}

% \section{Process view: workflows}
% \label{sec:Workflows}

LArSoft encompasses a collection of tools that can be roughly groups in
the following categories:

\begin{enumerate}
  \item simulation
  \item reconstruction
  \item analysis
  \item[\textbullet] display
\end{enumerate}

\begin{figure}[htbp]
\centering
\includegraphics[width=\textwidth]{figures/LArSoftArchitectureGraph.pdf}
\caption{\label{fig:LArSoftProcessingChain}Components of LArSoft and
their interaction with external libraries}
\end{figure}

The typical full processing chain (\cref{fig:LArSoftProcessingChain})
includes a reconstruction and an
analysis sequence. For simulated events, a preliminary simulation
sequence can be run.\\
Processing chains are defined by the experiments according to their
needs. LArSoft inherits the flexibility from the \emph{art} framework,
that provides users with the flexibility of choosing and arranging
processing modules at will, with the only limitation that each module
must be provided with all the information it needs to operate. The same
module can also be executed multiple times, with different
configurations.\\
The event display is capable of showing many of the data classes from
the simulation and reconstruction steps, and it includes a limited
ability of running modules with different configuration at run time.

LArSoft also provides an extensible set of data structures describing
objects involved in many levels of the physics analysis, e.g., the
time-dependent shape of signal from a photon detector, a simulated
neutrino or a reconstructed electromagnetic cascade. The use of these
common structures is key to flexibility, allowing to replace and
directly compare algorithms that use the same data structures.

\subsection{Simulation}\label{simulation}

The purpose of LArSoft simulation is to describe a realistic response of
the detectors to a known physics event (``truth''). Since the result of
the simulation should be equivalent to the output of the detectors, this
result is represented by the same data classes. The truth information,
not available from the detector, is produced and stored in additional
structures.

\begin{figure}[htbp]
\centering
\includegraphics[width=\textwidth]{figures/LArSoftSimulationGraph.pdf}
\caption{\label{fig:LArSoftsimulation}LArSoft simulation flow}
\end{figure}

The complete simulation chain is summarized in fig.
\ref{fig:LArSoftsimulation}. The process is typically described as three
steps:

\begin{enumerate}
\def\labelenumi{\arabic{enumi}.}
\item
  event generation
\item
  detector physics simulation
\item
  detector readout simulation
\end{enumerate}

The physics event can be generated by an external program or library.
LArSoft interfaces directly to GENIE generator (neutrino interactions)
and CRY (cosmic rays). It can also read a generic HEPEVT{[}ref{]}
format. In addition, LArSoft provides built-in generators to simulate
single particles, Argon nucleus decays, and more.

The detector physics simulation includes the interaction of the
generated particles with the detector, and the propagation to the
readout of produced photons and electrons. This part of the simulation
relies on GEANT4 for the interaction of particles with matter. Photon
and electron transportation to the readout are implemented in built-in
code. Detector parameters (e.g., the intensity of the electric field)
can be acquired from the job configuration or from a custom data base.

The last step transforms the physics information, electrons and photons,
into digitized detector response, including the simulation of
electronics noise and shaping. This is typically implemented with
experiment-specific code.

\subsection{Reconstruction}\label{reconstruction}

The reconstruction phase provides standard physics objects to describe
the physics event. Reconstruction delivers objects with different level
of sophistication and from different steps, as for example hits
describing localized charge deposition as detected on a wire, down to a
complete hierarchy of three-dimensional tracks. These objects are handed
over for further analysis.

\begin{figure}[htbp]
\centering
\includegraphics[width=\textwidth]{figures/LArSoftReconstructionGraph.pdf}
\caption{\label{fig:LArSoftReconstruction}LArSoft ``traditional''
reconstruction flow}
\end{figure}

Starting from detector response, either real or simulated, there are
many possible patterns of analysis. The more ``traditional'' one (fig.
\ref{fig:LArSoftReconstruction}) starts with the calibration of the
signals, attempting to suppress noise and revert electronics
distortions, and then it proceeds with the reconstruction of charge
deposition on a single TPC wire (\emph{hits}), to \emph{cluster} them in
groups lying on the same wire plane, and finally with combining clusters
from different planes in trajectories (\emph{tracks}) and particle
cascades (\emph{showers}), connected by interaction points
(\emph{vertices}). The hierarchal connection between them is called a
\emph{particle flow}. Many options are implemented in LArSoft for each
of these steps, that are interchangeable as they use the same input and
output classes.

During any of these steps the detector and data acquisition parameters
can be acquired from experiment data bases.

Any external library that utilizes LArSoft data classes to receive
inputs and deliver results is also fully interchangeable with the
algorithms implemented in LArSoft. A noticeable example is the
\emph{pandora} pattern recognition toolkit, that accepts LArSoft hits as
input and can present its results in the form of LArSoft clusters,
tracks and particle flow objects.

Further common analysis steps are the calibration of the energy
deposited in liquid argon by the interacting particles and their
identification as specific types (e.g., muons, protons, etc.).




\section{Deployment view: development and extensibility}
\label{sec:Development}

% \section{Deployment view: development and extensibility}
% \label{sec:Development}

The extensibility of LArSoft is largely based on the underlying
framework, \ART. The \ART framework processes physics \emph{event}
independently, executing on each of them a sequence of \emph{modules}.
An event is defined by an input module.
In most Experiments it is bound to a single pulsed beam interaction with the detector,
but test beam Experiments, non-beam Experiments and non-beam analyses (\eg proton decay)
may need to define different event boundaries.
The framework also provides a list of global \emph{services} that modules can rely on.
Examples of services implemented by LArSoft include
the description of detector geometry and channel mapping,
the set of detector configuration parameters,
and access to TPC channel quality information.

In this section we describe the development environment
and then focus on the main handles LArSoft offers developers for the sake of extensibility,
including new serializable data structures, new algorithms
and the use of external libraries.


\subsection{Development environment}
\label{ssec:Development:Environment}

LArSoft is designed for and supports the use of a development environment based on:
\begin{itemize}
   \item UNIX Product Support (\UPS) for access to dependent packages
   \item \cetbuildtools\cite{cetbuildtools} as build system
   \item Multi-Repository Build\cite{MRB} (\MRB)  to coordinate build and execute software from different repositories
   \item \git (recommended) or \SVN for version control
\end{itemize}
The following description assumes the prerequisite availability of all these tools.

LArSoft is fully supported on the following platforms:
\begin{itemize}
   \item Scientific Linux Fermi: version 6
   \item Darwin: version 13 (OS X 10.9 ``Maverick'') and 14 (OS X 10.10 ``Yosemite'')
\end{itemize}
LArSoft typically supports the two most recent versions of these operating systems\footnote{
The actual supported versions depend also on the underlying support of the O.S. by Fermilab.
}.
Support is also planned for the long term support release of Ubuntu Linux (16.04 LTS).

A typical workflow starts with the set up of a working area.
After the area is created the first time, subsequent utilization of it requires just a simple set up.
LArSoft provides a script for this set up,
and it is common practice for the Experiments to provide customized ones.

The development, whether it is creation of new code or modification of existing one,
follows the following workflow:
\begin{enumerate}
   \item \emph{development-specific set up} of the existing working area
   \item importing the source code to be modified, if any; this code will persist in the area
   \item modifications as needed
   \item building
   \item optional (and recommended) execution of a standard test suite
   \item installation for running
\end{enumerate}

The execution of LArSoft code including user development, as described above,
follows this workflow:
\begin{enumerate}
   \item \emph{run-time specific set up} of the existing working area
   \item preparation of job configuration as needed
   \item execution of the software
\end{enumerate}
The execution of LArSoft code as distributed, without modification,
has a simpler set up that does not require a development working area.\\
LArSoft and the Experiments provide a vast selection of configurations ready to run, making the second step optional.
Development and execution set up can coexist in the same environment at the same time.

LArSoft currently provides no facility to execute code remotely,
including job submission to remote clusters.
The Experiments supply workflows and scripts for this type of execution.



\subsection{Testing}
\label{ssec:Development:testing}

LArSoft development model allows multiple contributors to modify the
code at the same time. This model can create conflicts and dysfunction
in the code. Tests are instrumental to the early detection of such
defects. LArSoft includes tests at two levels, called \emph{unit tests}
and \emph{integration tests}.

Unit tests exercise a limited part of the system, typically a single algorithm.
Ideally a unit test for an algorithm should test all the functions of that algorithm.
In practice, tests for complex algorithms tend to set up and test a few known typical cases.
Unit tests can be added at the same time the tested code is beging developed.
They are run in the development environment:
as such, they are the quickest mean to exercise newly written code.

Integration tests involve the framework and one or more processing modules.
These tests can reproduce real user scenarios,
for example a part of the official processing chain of an Experiment,
and they can compare new and historical results.
LArSoft tools allow these tests to be run on demand at any time,
and a standard suite of tests is automatically and periodically run
as part of LArSoft Continuous Integration system.


\subsection{Data products}
\label{ssec:Development:DataProducts}

LArSoft provides a basic set of data structures.
Each structure is associated to a simple concept and a set of related quantities.
For example, \texttt{raw::RawDigit} describes the raw data as read from a TPC channel;
\texttt{recob::Cluster} describes a set of correlated hits observed on a wire plane;
\texttt{anab::Calorimetry} contains information about calibrated energy of a track.
A \emph{data product} is data that can be serialized and then recovered for further processing.
A data product can be:
\begin{itemize}
   \item a data structure
   \item a collection of data structures
   \item a set of associations between data structures
\end{itemize}

LArSoft strongly recommends data structures designed for serialization to follow some standard prescriptions:
\begin{itemize}
   \item simple: they contain just data, and trivial logic to access it;
      more complex elaborations belong to algorithms;
   \item concrete: they can be instantiated;
   \item containing data members from a restricted selection:
      \begin{itemize}
         \item fundamental C++ types (note that pointers are not fundamental);
         \item other types suitable themselves as data products
            (including collections as described below).
      \end{itemize}
\end{itemize}
Limitations of the ROOT I/O system impose restrictions on the types of allowed data members,
\eg on the set of supported C++11 containers.

Collections of data structures also undergo some prescriptions:
\begin{itemize}
   \item when contained as member of other data products,
      standard containers (from C++ STL) are preferred,
      favouring fixed-size arrays and STL vectors;
   \item when passed to the framework to be saved directly,
      STL vectors are strongly recommended;
   \item contained types must themselves be suitable as data products
      (as described above).
\end{itemize}

Relations between data products are expressed by \emph{associations}.
Associations are data products provided by \ART
that can relate a data product, or an element within a collection of data products,
to another data product or element.
Examples of use in LArSoft include associations between a
reconstructed hit and the calibrated signal it's reconstructed from, and
between a cluster and all the hits that constitute it.

Data products have a fundamental structural role:
they act as messages to be exchanged between algorithms.
As such, they are also the format in which most of the algorithm results are saved.
This allows to arbitrary split the processing chain in multiple sequences of jobs.


\subsection{User code}
\label{ssec:Development:UserCode}

Algorithms constitute, together with data products, the heart of LArSoft,
and the ability for the users to add their own algorithm is central to its design.
In fact, LArSoft algorithms differ from users' algorithms
only in the consideration that their purpose is of general interest.
Indeed, most of the algorithms in LArSoft were written by users to solve their own specific problems,
and then adopted into the common toolkit.
LArSoft encourages users to produce algorithms that perform correctly on \emph{any} liquid argon detector,
and to integrate them into LArSoft itself.

\begin{figure}
   \centering
   \includegraphics{figures/LArSoftFactorizationModelAndTests}
   \caption[LArSoft algorithm and service model]{
      \label{fig:AlgorithmModel}
      LArSoft algorithm and service model.
      Black lines represent ownership.
      The coloured arrows show the path the algorithm obtains the provider through.
      The green line contours the standard execution environment.
      Dotted lines describe testing environments:
      both service providers and algorithms can be tested without involving the full framework.
   }
\end{figure}

The preferred model for algorithm design is represented in \cref{fig:AlgorithmModel}.
We refer to this as \emph{factorization} model.
The underlying principle it is that the algorithm must be independently
testable and portable, using the minimal set of necessary dependences.
This also allows for the algorithms to be used in contexts where the
\ART framework is not available,
provided that some other system supplies equivalent functionalities as,
and only when, needed. The model is made of two layers:
\begin{enumerate}
   \item
      the algorithm, in the form of a class that
      \begin{itemize}
         \item
            is configurable with FHiCL parameter sets
         \item
            prefers LArSoft data products as input and output
         \item
            elaborates a single event or part of an event at a time
      \end{itemize}
   \item
      a module for the \ART framework, that:
      \begin{itemize}
         \item \label{response:0202.1}
           owns and manages the lifetime of one or more algorithm classes
         \item
           provides the algorithm(s) with the configuration, the data products
           and the information it needs to operate
         \item
           delivers algorithm output to the \ART framework
      \end{itemize}
\end{enumerate}
\label{response:0203.1}
LArSoft provides data products for many of the common concepts.
Algorithms that deal with these concepts should consume and produce such shared data products,
pairing them with additional, specific data structures
when more information needs to be carried around.

Since algorithms often rely on services, the services also need to
follow the same factorization model and be split in:

\begin{enumerate}
   \item
      a \emph{service provider}, in the form of a class that:
      \begin{itemize}
         \item
            is configurable with FHiCL parameter sets
         \item
            has the minimal convenient set of dependencies
         \item
            provides actual functionality
      \end{itemize}
   \item
      a service for the \ART framework, that:
      \begin{itemize}
         \item
            owns and manages the lifetime of its service provider
         \item
            provides modules with a pointer to the provider
         \item
            when relevant, reacts to messages from the framework (e.g., the
            beginning of a new run) and propagates them to the provider as needed
      \end{itemize}
\end{enumerate}
The module is also responsible of communicating to its algorithms which
service providers to use.
\label{response:0201.1}
The typical implementation of this pattern in the \ART framework is the following:
the module, when setting up the algorithm (possibly on each event),
obtains the services it needs from the framework,
obtains the provider from each of these services,
and uses the algorithm interface to propagate these providers to the algorithm itself.
\label{response:0205.1}
Algorithms exclusively interact with service providers rather than with \ART services.

Other important guidelines for the development of algorithms are:

\begin{description}
   \item[interoperability]
      they should document their assumptions in detail,
      and correctly perform on any detector if possible
   \item[modularity]
      each algorithm should perform a single task;
      complex tasks can be performed by hierarchies of algorithms
   \item[maintainability]
      they should come with complete documentation and proper tests
\end{description}

\Cref{fig:AlgorithmModel} shows that if algorithms are not
framework-dependent, their unit test can also be framework-independent.
Therefore, not only those algorithms can be developed in a simplified,
framework-unaware environment, but they can also be tested in that same
development environment. In other words, the full development cycle, of
which testing is an integral part, can seamlessly happen in the same
environment.


\subsection{External libraries}
\label{ssec:Development:ExternalLibraries}

We call ``external'' any library that does not depend on LArSoft, with
the possible exception of its data products. Examples in this category
are \GENIE, \GEANT, and \Pandora.

LArSoft's modularity can accommodate contributions from external
libraries into its workflow (\cref{fig:LArSoftAndExternals}). The
preferred way is to use directly the external library via its interface.
This requires an additional interface module between LArSoft and the
library, in charge of converting the LArSoft data products into a format
digestible by the external library, configuring and driving it, and
extracting and converting the results into a set of LArSoft data
products.
\begin{figure}
   \centering
   \includegraphics{figures/LArSoftAndExternalLibrary}
   \caption[Interaction between LArSoft and an external library]{
      \label{fig:LArSoftAndExternals}
      Interaction between LArSoft and an external library.
      The dashed line encompasses the components belonging to LArSoft.
      Shapes and colours are as in \cref{fig:LArSoftSimulationRelations}.
   }
\end{figure}

This model is exemplified in the interaction between LArSoft and
\Pandora\cite{pandora} (\cref{fig:LArSoftAndPandora}): \Pandora uses
its own data classes for input hits, particle flow results and geometry
specification. A base module exists that reads LArSoft hits, converts
them into \Pandora's, translates geometry information,
and LArSoft clusters, tracks, vertices, and more,
out of \Pandora particle flow objects.
\begin{figure}
   \centering
   \includegraphics{figures/LArSoftAndPandora}
   \caption[Model of communication between LArSoft and \Pandora]{
      \label{fig:LArSoftAndPandora}
      LArSoft workflow including \Pandora.
      Data structures in the \texttt{pandora} namespace are defined in \Pandora
      and also known by LArSoft.
   }
\end{figure}\\
This approach has relevant advantages: it can be fairly fast; it allows
a precise translation of information; it provides the greatest control
on the flow within the library; it defines and tracks the configuration
of the external library. Its greatest drawback is the need for the
LArSoft interface to depend on the external library. If this limitation
is not acceptable, a more independent communication channel can be
established via exchange files. In this case, LArSoft interface
translates data products into a neutral format, possibly based solely on
ROOT objects or on a textual representation, and back into data
products. The external library is in charge of performing the equivalent
operations with the library data format. This is for example the generic
communication mechanism with event generators that support HEPEVT
format. The strong decoupling comes at the price of a fragmented
execution chain and the burden of additional configuration consistency
control, for example to ensure that a consistent geometry was used for
the information (re)entering LArSoft.



\section{Physical view: repositories and packages}
\label{sec:Packages}

% \section{Deployment view: repositories and packages}
% \label{sec:Repositories}

LArSoft supports the use of the UNIX Product Support (\UPS) system
for deployment of LArSoft itself and of the additional software it depends from.
This system is organized in \emph{products} containing executable code for a specific platform
and auxiliary data as needed.
LArSoft set up demands from UPS a specific version of almost every library LArSoft depends on,
including for example the GNU compiler, Boost libraries and CERN ROOT.

LArSoft code base is organized in \emph{repositories} grouping different functionalities.
The current list of repositories is:
\begin{description}
   \item[larcore] independent of data products (\eg geometry description)
   \item[lardata] defining the shared data products
   \item[larevt] code independent of simulation and reconstruction algorithms (\eg calibration, database access)
   \item[larsim] detector simulation
   \item[larreco] physics object reconstruction
   \item[larana] depending on simulation or reconstruction algorithms (\eg particle identification, calorimetry)
   \item[larpandora] interface with pattern recognition package \Pandora
   \item[lareventdisplay] ROOT-based visualization tool
   \item[larexample] examples of LArSoft modules
   \item[larsoft] ``umbrella'' product
\end{description}
LArSoft repositories are maintained in Fermilab Redmine as \git repositories.


\subsection{Local LArSoft installation}
\label{ssec:Repositories:LocalInstallation}

LArSoft can be installed in any supported platform, either with:
\begin{description}
   \item[binary installation] copying prebuilt UPS products from Fermilab server
      into a local UPS directory
   \item[source installation] copying, building and installing into a local UPS directory
      the source code of each and every dependent package
\end{description}
Both installation patterns are supported via a single script.
In this way, LArSoft can be installed in virtual machines, personal computers
as well as in computing clusters.




\clearpage
\appendix

This stuff is going to be rearranged

\section{Architecture}\label{architecture}

\subsection{Overview}\label{overview}

LArSoft encompasses a collection of tools that can be roughly groups in
the following categories:

\begin{enumerate}
\def\labelenumi{\arabic{enumi}.}
\item
  simulation
\item
  reconstruction
\item
  analysis
\end{enumerate}

\begin{itemize}
\item
  display
\end{itemize}

\begin{figure}[htbp]
\centering
\includegraphics[width=\textwidth]{figures/LArSoftArchitectureGraph.pdf}
\caption{\label{fig:LArSoftProcessingChain}Components of LArSoft and
their interaction with external libraries}
\end{figure}

The typical full processing chain (fig.
\ref{fig:LArSoftProcessingChain}) includes a reconstruction and an
analysis sequence. For simulated events, a preliminary simulation
sequence can be run.\\
Processing chains are defined by the experiments according to their
needs. LArSoft inherits the flexibility from the \emph{art} framework,
that provides users with the flexibility of choosing and arranging
processing modules at will, with the only limitation that each module
must be provided with all the information it needs to operate. The same
module can also be executed multiple times, with different
configurations.\\
The event display is capable of showing many of the data classes from
the simulation and reconstruction steps, and it includes a limited
ability of running modules with different configuration at run time.

LArSoft also provides an extensible set of data structures describing
objects involved in many levels of the physics analysis, e.g., the
time-dependent shape of signal from a photon detector, a simulated
neutrino or a reconstructed electromagnetic cascade. The use of these
common structures is key to flexibility, allowing to replace and
directly compare algorithms that use the same data structures.

\subsection{Simulation}\label{simulation}

The purpose of LArSoft simulation is to describe a realistic response of
the detectors to a known physics event (``truth''). Since the result of
the simulation should be equivalent to the output of the detectors, this
result is represented by the same data classes. The truth information,
not available from the detector, is produced and stored in additional
structures.

\begin{figure}[htbp]
\centering
\includegraphics[width=\textwidth]{figures/LArSoftSimulationGraph.pdf}
\caption{\label{fig:LArSoftsimulation}LArSoft simulation flow}
\end{figure}

The complete simulation chain is summarized in fig.
\ref{fig:LArSoftsimulation}. The process is typically described as three
steps:

\begin{enumerate}
\def\labelenumi{\arabic{enumi}.}
\item
  event generation
\item
  detector physics simulation
\item
  detector readout simulation
\end{enumerate}

The physics event can be generated by an external program or library.
LArSoft interfaces directly to GENIE generator (neutrino interactions)
and CRY (cosmic rays). It can also read a generic HEPEVT{[}ref{]}
format. In addition, LArSoft provides built-in generators to simulate
single particles, Argon nucleus decays, and more.

The detector physics simulation includes the interaction of the
generated particles with the detector, and the propagation to the
readout of produced photons and electrons. This part of the simulation
relies on GEANT4 for the interaction of particles with matter. Photon
and electron transportation to the readout are implemented in built-in
code. Detector parameters (e.g., the intensity of the electric field)
can be acquired from the job configuration or from a custom data base.

The last step transforms the physics information, electrons and photons,
into digitized detector response, including the simulation of
electronics noise and shaping. This is typically implemented with
experiment-specific code.

\subsection{Reconstruction}\label{reconstruction}

The reconstruction phase provides standard physics objects to describe
the physics event. Reconstruction delivers objects with different level
of sophistication and from different steps, as for example hits
describing localized charge deposition as detected on a wire, down to a
complete hierarchy of three-dimensional tracks. These objects are handed
over for further analysis.

\begin{figure}[htbp]
\centering
\includegraphics[width=\textwidth]{figures/LArSoftReconstructionGraph.pdf}
\caption{\label{fig:LArSoftReconstruction}LArSoft ``traditional''
reconstruction flow}
\end{figure}

Starting from detector response, either real or simulated, there are
many possible patterns of analysis. The more ``traditional'' one (fig.
\ref{fig:LArSoftReconstruction}) starts with the calibration of the
signals, attempting to suppress noise and revert electronics
distortions, and then it proceeds with the reconstruction of charge
deposition on a single TPC wire (\emph{hits}), to \emph{cluster} them in
groups lying on the same wire plane, and finally with combining clusters
from different planes in trajectories (\emph{tracks}) and particle
cascades (\emph{showers}), connected by interaction points
(\emph{vertices}). The hierarchal connection between them is called a
\emph{particle flow}. Many options are implemented in LArSoft for each
of these steps, that are interchangeable as they use the same input and
output classes.

During any of these steps the detector and data acquisition parameters
can be acquired from experiment data bases.

Any external library that utilizes LArSoft data classes to receive
inputs and deliver results is also fully interchangeable with the
algorithms implemented in LArSoft. A noticeable example is the
\emph{pandora} pattern recognition toolkit, that accepts LArSoft hits as
input and can present its results in the form of LArSoft clusters,
tracks and particle flow objects.

Further common analysis steps are the calibration of the energy
deposited in liquid argon by the interacting particles and their
identification as specific types (e.g., muons, protons, etc.).

\subsection{Testing}\label{testing}

LArSoft development model allows multiple contributors to modify the
code at the same time. This model can create conflicts and dysfunction
in the code. Tests are instrumental to the early detection of such
defects. LArSoft includes tests at two levels, called \emph{unit tests}
and \emph{integration tests}.

Unit tests exercise a limited part of the system, typically a single
algorithm. Ideally a unit test for an algorithm should test all the
functions of that algorithm. In practice, tests for complex algorithms
tend to set up and test a few known typical cases.

Integration tests involve the framework and one or more processing
modules. These tests can reproduce real user scenarios, for example a
part of the official processing chain of an experiment, and they can
compare new and historical results. LArSoft tools allow these tests to
be run at any time, and a standard suite of tests is meant to be
automatically and periodically run.

\section{Extensibility}\label{extensibility}

The extensibility of LArSoft is largely based on the underlying
framework, \emph{art}. The \emph{art} framework processes physics event
independently, executing on each of them a sequence of modules. The
framework also provides a list of global ``services'' that modules can
rely on. Examples of services implemented by LArSoft include the
description of detector geometry and channel mapping, the set of
detector configuration parameters, and access to TPC channel quality
information.

Our description focuses on extensibility in terms of new persistable
data structures, of new algorithms implemented in LArSoft and of using
external libraries.

\subsection{Data products}\label{data-products}

LArSoft provides a basic set of persistable data classes. Each class is
associated to a simple concept and a set of related quantities. For
example, \texttt{raw::RawDigit} describes the raw data as read from a
TPC channel; \texttt{recob::Cluster} describes a set of hits observed on
a wire plane; \texttt{anab::Calorimetry} contains information about
calibrated energy of a track.

A \emph{data product} is a class that:

\begin{itemize}
\item
  is simple: contains just data and trivial logic to access it; more
  complex elaborations belong to algorithms
\item
  contains only members from a small selected libraries: C++ standard
  library is highly recommended; ROOT classes are also accepted
\item
  is not polymorphic
\end{itemize}

Limitations to ROOT I/O system impose restrictions on the types of
allowed data members, e.g., on the set of supported C++11 containers.
Relations between data products are expressed by \emph{associations}.
Associations are data products provided by \emph{art} that can relate a
data product, or an element of it, to another element from another data
product. Examples of use in LArSoft include the association between a
reconstructed hit and the calibrated signal it's reconstructed from, and
between a cluster and all the hits that constitute it.

Data products have a fundamental structural role: they act as messages
to be exchanged between algorithms. As such, they are also the format in
which most of the results are saved. This allows to arbitrary split the
processing chain in multiple sequences of jobs.

\subsection{User code}\label{user-code}

Algorithms constitute, together with data products, the heart of
LArSoft, and the ability for user to add their own algorithm is central
to its design. In fact, LArSoft algorithms differ from users' algorithms
only in the judgment that their purpose is considered of wider interest
than just for the single user. Indeed, most of the algorithms in LArSoft
were written by users to solve a specific problem, and then adopted into
the common toolkit. LArSoft encourages users to produce algorithms that
perform correctly on any liquid argon detector, and to integrate them
into LArSoft itself.

\begin{figure}[htbp]
\centering
\includegraphics[width=\textwidth]{figures/LArSoftSimplifiedFactorizationModel.pdf}
\caption{\label{fig:AlgorithmModel}LArSoft algorithm and service model}
\end{figure}

The preferred model for algorithm structure is represented in fig.
\ref{fig:AlgorithmModel}. We refer to it as \emph{factorization} model.
The underlying principle it is that the algorithm must be independently
testable and portable, using the minimal set of necessary dependences.
This also allows for the algorithms to be used in contexts where the
\emph{art} framework is not available, provided that some other system
supplies equivalent functionalities as, and only when, needed. The model
is made of two layers:

\begin{enumerate}
\def\labelenumi{\arabic{enumi}.}
\item
  the algorithm, in the form of a class that
\end{enumerate}

\begin{itemize}
\item
  is configurable with FHiCL parameter sets
\item
  consumes LArSoft data products as input
\item
  produces LArSoft data products as output
\item
  has the minimal convenient set of dependencies
\item
  elaborates a single event or part of an event at a time
\end{itemize}

\begin{enumerate}
\def\labelenumi{\arabic{enumi}.}
\setcounter{enumi}{1}
\item
  a module for the \emph{art} framework, that:
\end{enumerate}

\begin{itemize}
\item
  owns and manages the lifetime of one or more algorithm classes
\item
  provides the algorithm(s) with the configuration, the data products
  and the information it needs to operate
\item
  delivers algorithm output to the \emph{art} framework
\end{itemize}

Since algorithms often rely on services, the services also need to
follow the same factorization model and be split in:

\begin{enumerate}
\def\labelenumi{\arabic{enumi}.}
\item
  a \emph{service provider}, in the form of a class that:
\end{enumerate}

\begin{itemize}
\item
  is configurable with FHiCL parameter sets
\item
  has the minimal convenient set of dependencies
\item
  provides the actual functionalities
\end{itemize}

\begin{enumerate}
\def\labelenumi{\arabic{enumi}.}
\setcounter{enumi}{1}
\item
  a service for the \emph{art} framework, that:
\end{enumerate}

\begin{itemize}
\item
  owns and manages the lifetime of its service provider
\item
  provides modules with a pointer to the provider
\item
  when relevant, propagates messages from the framework (e.g., the
  beginning of a new run) to the provider
\end{itemize}

The module is also responsible of communicating to its algorithms which
service providers to use. Algorithms exclusively interact with service
providers rather than with \emph{art} services.

Other important guidelines for the development of algorithms are:

\begin{itemize}
\item
  interoperability: they should document their assumptions in detail,
  and correctly perform on any detector if possible
\item
  modularity: each algorithm should perform a single task; complex tasks
  can be performed by hierarchies of algorithms
\item
  maintainability: they should come with complete documentation and
  proper tests
\end{itemize}

Figure \ref{fig:AlgorithmModel} shows that if algorithms are not
framework-dependent, their unit test can also be framework-independent.
Therefore, not only those algorithms can be developed in a simplified,
framework-unaware environment, but they can also be tested in that same
development environment. In other words, the full development cycle, of
which testing is an integral part, can seamlessly happen in the same
environment.

\subsection{External libraries}\label{external-libraries}

We call ``external'' any library that does not depend on LArSoft, with
the possible exception of its data products. Examples in this category
are GENIE, GEANT4, and \emph{pandora}.

\begin{figure}[htbp]
\centering
\includegraphics[width=\textwidth]{figures/LArSoftAndExternals.pdf}
\caption{\label{fig:LArSoftAndExternals}Interaction between LArSoft and
an external library}
\end{figure}

LArSoft's modularity can accommodate contributions from external
libraries into its workflow (fig. \ref{fig:LArSoftAndExternals}). The
preferred way is to use directly the external library via its interface.
This requires an additional interface module between LArSoft and the
library, in charge of converting the LArSoft data products into a format
digestible by the external library, configuring and driving it, and
extracting and converting the results into a set of LArSoft data
products.

\begin{figure}[htbp]
\centering
\includegraphics[width=\textwidth]{figures/LArSoftAndPandora.pdf}
\caption{\label{fig:LArSoftAndPandora}Interaction between LArSoft and
\emph{pandora}}
\end{figure}

This is exemplified in the interaction between LArSoft and
\emph{pandora} (fig. \ref{fig:LArSoftAndPandora}): \emph{pandora} uses
its own data classes for input hits, particle flow results and geometry
specification. A base module exists that reads LArSoft hits, converts
them into \emph{pandora}'s, translates geometry information, and
recreates out of \emph{pandora} particle flow objects LArSoft clusters,
tracks, vertices, and more.

This approach has relevant advantages: it can be fairly fast; it allows
a precise translation of information; it provides the greatest control
on the flow within the library; it defines and tracks the configuration
of the external library. Its greatest drawback is the need for the
LArSoft interface to depend on the external library. If this limitation
is not acceptable, a more independent communication channel can be
established via exchange files. In this case, LArSoft interface
translates data products into a neutral format, possibly based solely on
ROOT objects or on a textual representation, and back into data
products. The external library is in charge of performing the equivalent
operations with the library data format. This is for example the generic
communication mechanism with event generators that support HEPEVT
format. The strong decoupling comes at the price of a fragmented
execution chain and the burden of additional configuration consistency
control, for example to ensure that a consistent geometry was used for
the information (re)entering LArSoft.


%%%%%%%%%%%%%%%%%%%%%%%%%%%%%%%%%%%%%%%%%%%%%%%%%%%%%%%%%%%%%%%%%%%%%%%%%%%%%%%%
%%% References


% the order of references is determined by the order of \cite and \nocite commands
\bibliographystyle{ieeetr}

\bibliography{sources/references}




\clearpage

\section*{Comments}\label{comments}

\subsection*{\texorpdfstring{Sunday 7:51:39, Ruth Pordes
\href{mailto:ruth@fnal.gov}{\nolinkurl{ruth@fnal.gov}} (``Re: First
draft of the architecture description
document'')}{Sunday 7:51:39, Ruth Pordes ruth@fnal.gov (``Re: First draft of the architecture description document'')}}\label{sunday-75139-ruth-pordes-ruthfnal.gov-re-first-draft-of-the-architecture-description-document}

\emph{{[}Q 001{]}} does the scope include human interfaces as well as
software?\\
\emph{{[}A 001.1{]}} \emph{{[}GP{]}} I thought mostly not, but I am not
completely sure what human interface includes. To be clarified.
\emph{(TODO)}

\emph{{[}Q 002{]}} nutools event display facility and simulation data
structures -- still does not make sense to me. Is Visualization one
special kind of analysis or does Larsoft have specific interfaces to
it?\\
\emph{{[}A 002.1{]}} \emph{{[}GP{]}} Visualization is a special kind of
analysis. But our event display crosses the border with its (limited)
ability to \emph{interactively} reprocess the input.

\emph{{[}Q 003{]}} page 4 -- can components of the chain be re-executed
during a single pass?-\\
\emph{{[}A 003.1{]}} \emph{{[}GP{]}} I have added a couple of sentences
in the previous-to-last paragraph of Architecture \textgreater{}
Overview section, that I hope give an answer. The answer is very much in
the features of art, that I have not covered at all in this text. Should
we? \emph{(TODO)}

\emph{{[}Q 004{]}} does event display have a specific meaning - I'll
include it in the Requirements glossary -- it is different from a
generalized visualization and I presume the definition should explain
this? Also, if the event display is in nutools it is not part of
larsoft??? Can we share a glossary in some fashion?\\
\emph{{[}A 004.1{]}} \emph{{[}GP{]}}

\emph{{[}Q 005{]}} Figure 2. You explicitly mean Detector not DAQ ?
Does/shoud daq show up somewhere\\
\emph{{[}A 005.1{]}} \emph{{[}GP{]}} in practice DAQ products is what we
communicate with. It doesn't have to be only that, but I guess that is
it effectively what happens. \emph{(TODO)}

\emph{{[}Q 006{]}} a Fluka interface is in the works with integration
hoped for before the end of Dec. Can you include a sentence on this
interface?\\
\emph{{[}A 006.1{]}} \emph{{[}GP{]}} Erica, confirm? \emph{(TODO)}

\emph{{[}Q 007{]}} page 10.Unit test. These are important. These are not
the only tests. I don't see them referred to and perhaps some more
specifics might be useful?\\
\emph{{[}A 007.1{]}} \emph{{[}GP{]}} I added a section about testing. I
have added a few words also at the point Ruth specified (at the end of
``User code'' section). I think it would be good to add a ``test'' block
in one of the high-level diagrams, but I can't figure out where
(probably in \ref{fig:LArSoftProcessingChain}, but how?). Or maybe we
have to add a \emph{development model} section?



\end{document}
