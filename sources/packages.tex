% \section{Deployment view: repositories and packages}
% \label{sec:Repositories}

LArSoft supports the use of the UNIX Product Support (\UPS) system
for deployment of LArSoft itself and of the additional software it depends from.
This system is organized in \emph{products} containing executable code for a specific platform
and auxiliary data as needed.
LArSoft set up demands from UPS a specific version of almost every library LArSoft depends on,
including for example the GNU compiler, Boost libraries and CERN ROOT.

LArSoft code base is organized in \emph{repositories} grouping different functionalities.
The current list of repositories is:
\begin{description}
   \item[larcore] independent of data products (\eg geometry description)
   \item[lardata] defining the shared data products
   \item[larevt] code independent of simulation and reconstruction algorithms (\eg calibration, database access)
   \item[larsim] detector simulation
   \item[larreco] physics object reconstruction
   \item[larana] depending on simulation or reconstruction algorithms (\eg particle identification, calorimetry)
   \item[larpandora] interface with pattern recognition package \Pandora
   \item[lareventdisplay] ROOT-based visualization tool
   \item[larexample] examples of LArSoft modules
   \item[larsoft] ``umbrella'' product
\end{description}
LArSoft repositories are maintained in Fermilab Redmine as \git repositories.


\subsection{Local LArSoft installation}
\label{ssec:Repositories:LocalInstallation}

LArSoft can be installed in any supported platform, either with:
\begin{description}
   \item[binary installation] copying prebuilt UPS products from Fermilab server
      into a local UPS directory
   \item[source installation] copying, building and installing into a local UPS directory
      the source code of each and every dependent package
\end{description}
Both installation patterns are supported via a single script.
In this way, LArSoft can be installed in virtual machines, personal computers
as well as in computing clusters.
