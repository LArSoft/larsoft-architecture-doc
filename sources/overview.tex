% \section{Overview}
% \label{sec:Overview}

% 
% this is my reinterpretation of the "use case view" suggested by Saba
% 

The LArSoft toolkit aims to offer a solution for the typical data analysis scenarios
of an experiment based on a Liquid Argon TPC detector:
\begin{itemize}
  \item generation of physics pseudo-events
  \item simulation of physics processes in the detectors
  \item simulation of the readout response
  \item reconstruction of low and high level physics objects
  \item analysis and presentation of collected data
  \item graphical display of physics events
\end{itemize}

As an example, suppose a scientist may want to develop a new clustering algorithm
optimized for a certain type of physics events.
LArSoft offers interface to generators to produce either simplified physics events
or more realistic ones that include, for example, cosmic radiation.
It also provides the simulation of those events in the specific experiment detector.
If the target processes are common enough, the experiment might have already
executed these steps on large scale, also using LArSoft, and provided the necessary input.
The scientist is then presented with standard interfaces to access geometry and detector
information, and standard data structure to start the reconstruction with,
including single wire hits suitable as starting point for a clustering algorithm,
and to store the results into.
She (or he) can use the standard framework environment to write an algorithm class
and its framework module, compile it and test it immediately on simulated data.
The result, in a standard LArSoft structure, can be immediately visualized in a
event display, and improvement made to the code.
Depending on the algorithm, the time between a code change and the
visualization of its effect may take less than one minute.
Finally, replacing the input with actual detector data, that uses the same format
as the simulation, she will immediately see the performance in the real case.

As a different example, a scientist may want to compare two different algorithms
analyzing reconstructed tracks.
After the tracks are produced, either running the track reconstruction algorithm
on simulated or real data, he (or she) will write one or more analysis algorithms
and their framework modules to produce the necessary plots.

Since LArSoft has been designed to take advantage of the \ART framework\cite{ART},
LArSoft users will extensively work with its concepts,
including for example services and modules, and infrastructure, like \ART scripts
to create skeleton modules, and the configuration based on FHiCL language\cite{FHiCL}.

