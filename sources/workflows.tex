% \section{Process view: workflows}
% \label{sec:Workflows}

LArSoft encompasses a collection of tools that can be roughly groups in
the following categories:

\begin{enumerate}
  \item simulation
  \item reconstruction
  \item analysis
  \item[\textbullet] display
\end{enumerate}

\begin{figure}[htbp]
\centering
\includegraphics[width=\textwidth]{figures/LArSoftArchitectureGraph.pdf}
\caption{\label{fig:LArSoftProcessingChain}Components of LArSoft and
their interaction with external libraries}
\end{figure}

The typical full processing chain (\cref{fig:LArSoftProcessingChain})
includes a reconstruction and an
analysis sequence. For simulated events, a preliminary simulation
sequence can be run.\\
Processing chains are defined by the experiments according to their
needs. LArSoft inherits the flexibility from the \emph{art} framework,
that provides users with the flexibility of choosing and arranging
processing modules at will, with the only limitation that each module
must be provided with all the information it needs to operate. The same
module can also be executed multiple times, with different
configurations.\\
The event display is capable of showing many of the data classes from
the simulation and reconstruction steps, and it includes a limited
ability of running modules with different configuration at run time.

LArSoft also provides an extensible set of data structures describing
objects involved in many levels of the physics analysis, e.g., the
time-dependent shape of signal from a photon detector, a simulated
neutrino or a reconstructed electromagnetic cascade. The use of these
common structures is key to flexibility, allowing to replace and
directly compare algorithms that use the same data structures.

\subsection{Simulation}\label{simulation}

The purpose of LArSoft simulation is to describe a realistic response of
the detectors to a known physics event (``truth''). Since the result of
the simulation should be equivalent to the output of the detectors, this
result is represented by the same data classes. The truth information,
not available from the detector, is produced and stored in additional
structures.

\begin{figure}[htbp]
\centering
\includegraphics[width=\textwidth]{figures/LArSoftSimulationGraph.pdf}
\caption{\label{fig:LArSoftsimulation}LArSoft simulation flow}
\end{figure}

The complete simulation chain is summarized in fig.
\ref{fig:LArSoftsimulation}. The process is typically described as three
steps:

\begin{enumerate}
\def\labelenumi{\arabic{enumi}.}
\item
  event generation
\item
  detector physics simulation
\item
  detector readout simulation
\end{enumerate}

The physics event can be generated by an external program or library.
LArSoft interfaces directly to GENIE generator (neutrino interactions)
and CRY (cosmic rays). It can also read a generic HEPEVT{[}ref{]}
format. In addition, LArSoft provides built-in generators to simulate
single particles, Argon nucleus decays, and more.

The detector physics simulation includes the interaction of the
generated particles with the detector, and the propagation to the
readout of produced photons and electrons. This part of the simulation
relies on GEANT4 for the interaction of particles with matter. Photon
and electron transportation to the readout are implemented in built-in
code. Detector parameters (e.g., the intensity of the electric field)
can be acquired from the job configuration or from a custom data base.

The last step transforms the physics information, electrons and photons,
into digitized detector response, including the simulation of
electronics noise and shaping. This is typically implemented with
experiment-specific code.

\subsection{Reconstruction}\label{reconstruction}

The reconstruction phase provides standard physics objects to describe
the physics event. Reconstruction delivers objects with different level
of sophistication and from different steps, as for example hits
describing localized charge deposition as detected on a wire, down to a
complete hierarchy of three-dimensional tracks. These objects are handed
over for further analysis.

\begin{figure}[htbp]
\centering
\includegraphics[width=\textwidth]{figures/LArSoftReconstructionGraph.pdf}
\caption{\label{fig:LArSoftReconstruction}LArSoft ``traditional''
reconstruction flow}
\end{figure}

Starting from detector response, either real or simulated, there are
many possible patterns of analysis. The more ``traditional'' one (fig.
\ref{fig:LArSoftReconstruction}) starts with the calibration of the
signals, attempting to suppress noise and revert electronics
distortions, and then it proceeds with the reconstruction of charge
deposition on a single TPC wire (\emph{hits}), to \emph{cluster} them in
groups lying on the same wire plane, and finally with combining clusters
from different planes in trajectories (\emph{tracks}) and particle
cascades (\emph{showers}), connected by interaction points
(\emph{vertices}). The hierarchal connection between them is called a
\emph{particle flow}. Many options are implemented in LArSoft for each
of these steps, that are interchangeable as they use the same input and
output classes.

During any of these steps the detector and data acquisition parameters
can be acquired from experiment data bases.

Any external library that utilizes LArSoft data classes to receive
inputs and deliver results is also fully interchangeable with the
algorithms implemented in LArSoft. A noticeable example is the
\emph{pandora} pattern recognition toolkit, that accepts LArSoft hits as
input and can present its results in the form of LArSoft clusters,
tracks and particle flow objects.

Further common analysis steps are the calibration of the energy
deposited in liquid argon by the interacting particles and their
identification as specific types (e.g., muons, protons, etc.).

