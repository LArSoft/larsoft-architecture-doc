% \section{Process view: workflows}
% \label{sec:Workflows}

LArSoft tools can be sequenced and combined to compound complete workflows.
The typical usage is aligned to three main types of ``standard'' workflows: % (\cref{fig:LArSoftProcessingChain}):
\begin{enumerate}
  \item simulation
  \item reconstruction
  \item analysis
\end{enumerate}
where the first step, simulation, is of course skipped when processing real detector data.
% \begin{figure}[htbp]
%   \centering
%   \includegraphics[width=\textwidth]{figures/LArSoftWorkflows.pdf}
%   \caption{\label{fig:LArSoftProcessingChain}Typical workflow sequence in liquid argon TPC analysis. Each block represents a workflow.}
%   % REMOVE DIAGRAM
% \end{figure}
LArSoft does not directly define these processing chains.
Rather, it inherits the flexibility from the \emph{art} framework,
which provides users with the flexibility of choosing and arranging processing modules at will.
%, with the only limitation that each module
% must be provided with all the information it needs to operate.
% A module can be executed multiple times, with different configurations.\\
Thus, the Experiments define the steps of each workflow according to their needs.
Still, these needs are fairly shared,
and it is possible to characterize a ``typical'' chain for each workflow.


\subsection{Simulation workflow}
\label{ssec:Workflows:Simulation}

The purpose of a simulation workflow is to describe a realistic response of
the detectors to a known physics event (``truth'').
Since the result of the simulation should be equivalent to the output of the detectors,
this result is represented by the same data classes.

The main results of this workflow are:
\begin{itemize}
	\item data products representing the detector response (\eg \ClassName{raw::RawDigit} and \ClassName{raw::OpDetWaveform})
	\item data products representing the simulated physics (\eg \ClassName{simb::MCTruth}, \ClassName{simb::MCParticle}
\end{itemize}

\begin{figure}[htbp]
  \centering
  \includegraphics{figures/LArSoftSimulationWorkflow.pdf}
  \caption{\label{fig:LArSoftSimulation}A typical LArSoft simulation workflow.}
\end{figure}
The complete simulation chain is summarized in \cref{fig:LArSoftSimulation}.
The process is typically divided in three steps:
\begin{enumerate}
   \item event generation
   \item detector physics simulation
   \item detector readout simulation
      \begin{itemize}
         \item TPC: signal on the wires
         \item optical detector
         \item ``auxiliary'' detectors (\eg scintillator pads)
      \end{itemize}
\end{enumerate}

The physics event can be generated by an external program or library.
LArSoft currently interfaces directly to GENIE generator (neutrino interactions)
and CRY (cosmic rays). It can also read events stored in HEPEVT\cite{HEPEVT} format.
In addition, LArSoft provides built-in generators for single particles,
Argon nucleus decays, and more.

The detector physics simulation includes the interaction of the
generated particles with the detector,
and the transportation to the readout of produced photons and electrons.
This part of the simulation currently relies on GEANT4 for the interaction of particles with matter.
Photon and electron transportation are implemented in built-in code.
Detector parameters (\eg, the intensity of the electric field)
can be acquired from the job configuration or from a custom data base.

The last step transforms the physics information, electrons and photons,
into digitized detector response,
including the simulation of electronics noise and shaping.
This is typically implemented with experiment-specific code,
and separately for each detector type.


\subsection{Reconstruction workflow}
\label{ssec:Workflows:Reconstruction}

The reconstruction phase produces standard physics objects describing the physics event.
Reconstruction delivers objects with different level of sophistication,
as for example hits describing localized charge deposition as detected on a wire,
down to a complete hierarchy of three-dimensional tracks.
These objects are handed over for further analysis.
\begin{figure}
   \centering
   \includegraphics{figures/LArSoftReconstructionWorkflow.pdf}
   \caption{\label{fig:LArSoftReconstruction}A typical LArSoft reconstruction flow.}
\end{figure}
Through the workflow, detector and data acquisition parameters
can be acquired from Experiment data bases.

Many possible reconstruction strategies are possible.
LArSoft allows them to be applied indifferently to data produced by a real detector or simulated.
The more ``traditional'' one \cref{fig:LArSoftReconstruction}) proceeds through:
\begin{enumerate}
  \item calibration of the signals, noise suppression and removal of electronics distortions;
  \item independent reconstruction of charge deposition on each TPC wire (\emph{hits});
  \item definition of \emph{clusters} from hits lying on the same wire plane;
  \item combination of clusters from different planes in trajectories (\emph{tracks}) and particle cascades (\emph{showers});
  \item identification of interaction points (\emph{vertices});
  \item hierarchal connection of them into \emph{particle flow} structures. Many options are implemented in LArSoft for each.
\end{enumerate}
Different algorithms can be chosen to perform each of these steps.
Any external library that utilizes LArSoft data classes to receive inputs
and deliver results is also fully interchangeable with the algorithms implemented in LArSoft.
A noticeable example is the \emph{pandora} pattern recognition toolkit,
that accepts LArSoft hits as input and can present its results in the form of LArSoft clusters,
tracks and particle flow objects.

Alternative workflows can and have been developed.
For example, a derivation of the workflow described above consists in a complete
first pass tuned to the reconstruction and subsequent identification of background
objects (mostly cosmic rays), in their elimination at the level of hits,
and a second pass tuned for reconstruction of neutrino interactions.

Other approaches include direct track reconstruction without clustering;
direct clustering from calibrated or uncalibrated channel signals by image processing algorithms,
bypassing the construction of hits;
perform a quick track reconstruction
and use the result to better direct the algorithms during a second pass;
and more.

LArSoft and \ART modularity allows to arrange for acyclic workflows
with any predetermined number of (potentially optional) steps.
It does not accommodate cyclic workflows.


\subsection{Analysis workflow}
\label{ssec:Workflows:Analysis}

Analysis workflows are the most vaguely defined,
due in part to the more diverse goals,
and partly to the fact that in this relatively early stage the Experiments
have devoted most of the time to simulation and reconstruction.

The calibration of energy deposited in liquid argon by interacting particles
and their identification as specific types (\eg, muons, protons, etc.)
have been classified sometimes as ``analysis'', sometimes as ``reconstruction''.
Another common analysis task is evaluation of reconstruction performances
and comparison between different algorithms and strategies.

Calibration activities, for example pedestal analysis,
characterization of argon purity, mapping of the electric field,
also fall in this category and they are ideal candidates for the standardization of workflows.
