% \section{Process view: workflows}
% \label{sec:Workflows}

LArSoft tools can be sequenced and combined to define complete workflows.
The typical usage is aligned to three main types of ``standard'' workflows: % (\cref{fig:LArSoftProcessingChain}):
simulation, reconstruction and analysis.
% \begin{enumerate}
%    \item simulation
%    \item reconstruction
%    \item analysis
% \end{enumerate}
Simulation step is, of course, skipped when processing real detector data.
% \begin{figure}
%   \centering
%   \includegraphics[width=\textwidth]{figures/LArSoftWorkflows}
%   \caption{\label{fig:LArSoftProcessingChain}Typical workflow sequence in liquid argon TPC analysis. Each block represents a workflow.}
%   % REMOVE DIAGRAM
% \end{figure}
LArSoft does not directly define these processing chains.
Rather, it relies on the flexibility of \emph{art} framework,
that provides users with the flexibility of choosing and arranging processing modules at will.
%, with the only limitation that each module
% must be provided with all the information it needs to operate.
% A module can be executed multiple times, with different configurations.\\
Thus, the Experiments define the steps of each workflow according to their needs.
Nevertheless, experience has shown that,
since the same needs being shared among Experiments,
it is possible to characterize ``typical'' workflow chains.


\subsection{Simulation workflow}
\label{ssec:Workflows:Simulation}

The purpose of a simulation workflow is to describe a realistic response of
the detectors to a known physics event (``Monte Carlo truth'' or simply ``truth'').
Since the result of the simulation should be equivalent to the output of the detectors,
the same data structures represent both simulation results and data from the detectors.

The main products of this workflow are:
\begin{itemize}
   \item data products representing the detector response (\eg \ClassName{raw::RawDigit} and \ClassName{raw::OpDetWaveform})
   \item data products representing the simulated physics (\eg \ClassName{simb::MCTruth}, \ClassName{simb::MCParticle})
\end{itemize}

\begin{figure}
   \centering
   \includegraphics{figures/LArSoftSimulationWorkflow}
   \caption[Typical LArSoft simulation workflow]{\label{fig:LArSoftSimulationWorkflow}
      A typical LArSoft simulation workflow.
      Blocks represent operations and ellipses represent data.
      All data can be saved in LArSoft format for later use.
      Block colour coding follows \cref{fig:LArSoftRelations}.
   }
\end{figure}
The complete simulation chain is illustrated in \cref{fig:LArSoftSimulationWorkflow}.
The process is typically divided in three steps:
\begin{enumerate}
   \item physics event generation
   \item detector physics simulation
   \item detector readout simulation
      \begin{itemize}
         \item TPC (signal on the wires)
         \item optical detector(s)
         \item ``auxiliary'' detectors (\eg scintillator pads)
      \end{itemize}
\end{enumerate}

\begin{figure}
   \centering
   \includegraphics{figures/LArSoftEventGeneration}
   \caption[Event generation workflow]{\label{fig:LArSoftEventGeneration}
      Event generation workflow.
      Colour coding follows \cref{fig:LArSoftSimulationWorkflow}.
      Purple lines describe data flow. The black line represents a relation via API.
      The dashed line encompasses the part of the workflow driven via LArSoft configuration.
   }
\end{figure}
The physics event can be generated by an external program or library (\cref{fig:LArSoftEventGeneration}).
LArSoft can interface directly to generators that offer API, through a specific driver module.
Currently driver modules are offered to work
with GENIE generator (neutrino interactions) and CRY (cosmic rays).
LArSoft can also read events stored in HEPEVT format.
In addition, LArSoft provides built-in generators for single particles,
Argon nucleus decays, and more.

\begin{figure}
   \centering
   \includegraphics{figures/LArSoftSimulationComponents}
   \caption[Relation between components for a typical reconstruction workflow]{
      \label{fig:LArSoftSimulationRelations}
      Components involved in simulation and their relation.
      Blocks represent operations and ellipse represents data.
      Lines represent communication between components: via data transfer (purple lines) and via API (black lines). 
      Block colour coding follows \cref{fig:LArSoftRelations}.
      The encircled area maps the simulation workflow illustrated in \cref{fig:LArSoftSimulationWorkflow}.
   }
\end{figure}
The detector physics simulation includes interaction of 
generated particles with the detector,
and transportation to the readout of produced photons and electrons.
This part of the simulation currently relies on \GEANT for the interaction of particles with matter.
Electron and photon transportation is implemented in built-in code.
To make photon transportation faster, Experiments can fill and use maps
that parametrize the exposure of each optical subdetector
as function of the scintillation position in the cryostat.
Detector parameters (\eg, the intensity of the electric field)
can be acquired from the job configuration or from a Experiment database.

The last step transforms the physics information, electrons and photons,
into digitized detector response,
including the simulation of electronics noise and shaping.
This is typically implemented with experiment-specific code,
and separately for each sub-detector type.


\subsection{Reconstruction workflow}
\label{ssec:Workflows:Reconstruction}

The reconstruction phase produces standard physics objects describing the physics event.
Reconstruction delivers objects with different level of sophistication,
as for example hits describing localized charge deposition as detected on a wire,
down to a complete hierarchy of three-dimensional tracks.
These objects can be permanently stored for further analysis.
\begin{figure}
   \centering
   \includegraphics{figures/LArSoftReconstructionWorkflow}
   \caption[Typical LArSoft reconstruction workflow]{
      \label{fig:LArSoftReconstructionWorkflow}
      A typical LArSoft reconstruction workflow.
      The area within the dashed contour highlights the physics objects reconstruction.
      Shapes and colours are as in \cref{fig:LArSoftSimulationWorkflow}.
   }
\end{figure}
Through the workflow, detector and data acquisition parameters
can be acquired from Experiment databases.

Many possible reconstruction strategies are possible.
LArSoft allows them to be applied indifferently to data produced by a real detector or simulated.
Support is planned for mixing the two types of sources together.
The more ``traditional'' strategy (\cref{fig:LArSoftReconstructionWorkflow}) proceeds through:
\begin{enumerate}
  \item calibration of the signals, noise suppression and removal of electronics distortions;
  \item reconstruction of charge deposition independently on each TPC wire (\emph{hits});
  \item definition of \emph{clusters} from hits lying on the same wire plane;
  \item combination of clusters from different planes in trajectories (\emph{tracks}) and particle cascades (\emph{showers});
  \item identification of interaction points (\emph{vertices});
  \item hierarchal connection of them into \emph{particle flow} structures. Many options are implemented in LArSoft for each.
\end{enumerate}
Different algorithms can be chosen to perform each of these steps.
Any external library that utilizes LArSoft data classes to receive inputs
and deliver results is also fully interchangeable with the algorithms implemented in LArSoft.
A noticeable example is the \Pandora pattern recognition toolkit,
that accepts LArSoft hits as input and can present its results in the form of LArSoft clusters,
tracks and particle flow objects.

\begin{figure}
   \centering
   \includegraphics{figures/LArSoftReconstructionRelations}
   \caption[Relation between components for a typical reconstruction workflow]{
      \label{fig:LArSoftReconstructionComponents}
      Components involved in a typical reconstruction workflow and their relation.
      Shapes and colours are as in \cref{fig:LArSoftSimulationRelations}.
   }
\end{figure}

This workflow is extremely simple and factorizes the process in highly independent steps.
Alternative workflows can and have been developed.\\
Examples include a workflow derived from the one described above,
the first pass of which consists of a complete reconstruction
tuned for the identification of background objects (mostly cosmic rays),
followed by elimination of the identified background % at the level of hits
and by a second pass tuned for reconstruction of neutrino interactions.\\
Other solutions include direct track reconstruction without clustering;
direct clustering from calibrated or uncalibrated channel signals by image processing algorithms,
bypassing the construction of hits;
a quick, preliminary track reconstruction
followed by a complete reconstruction utilizing the first result to better direct the algorithms;
inverting the order of track and vertex finding;
and more.\\
Any number of object collections can exist at the same time,
and these alternative approaches can be executed in the same job,
and their results saved in the same file and accessed as needed.

LArSoft and \ART modularity supports any acyclic workflows,
with any predetermined number of (potentially optional) steps.
It does not accommodate cyclic workflows.


\subsection{Analysis workflow}
\label{ssec:Workflows:Analysis}

Analysis workflows are the most vaguely defined,
due in part to the more diverse goals,
and partly to the fact that in this relatively early stage the Experiments
have devoted most of the time to simulation and reconstruction.
No prototype analysis workflow has emerged yet.

The calibration of energy deposited in liquid argon by interacting particles
and their identification as specific types --- \eg, muons, protons, etc. ---
 (see the end of the reconstruction workflow in \cref{fig:LArSoftReconstructionWorkflow})
have been considered sometimes as part of the analysis, sometimes as part of the reconstruction.
Another common analysis task is evaluation of reconstruction performances
and comparison between different algorithms and strategies.

Calibration activities, for example pedestal analysis,
characterization of argon purity, mapping of the electric field,
also fall in this category and they are ideal candidates for the standardization of workflows.
