% \section{Introduction}
% \label{sec:introduction}

\subsection{Purpose}
\label{ssec:Introduction:Purpose}

The LArSoft toolkit enables simulation, reconstruction and
physics analysis of data from any detection system based on Liquid Argon TPCs.
Its common tools and algorithms render the development
and analysis process more uniform across the Experiments,
and facilitate direct sharing of code and experience between Experiments.
LArSoft is extensible to accommodate evolving Experiments' needs
and adoption by new Experiments.

This document describes the current architecture of LArSoft toolkit.
The architecture was developed according to, and therefore reflects,
the consensus of LArSoft partners, including the adopting Experiments.

The document provides a reference for the reader interested in learning
the general structure of LArSoft, its functional areas and interactions
with the execution environment.
It also offers guidelines for the contributor aiming to develop new algorithms
within LArSoft or to use it together with external tools.


\subsection{Scope}
\label{ssec:Introduction:Scope}

This document provides an overview of the architecture of LArSoft toolkit,
including its relationship with the surrounding software environment.
The internal flow of the different subsystems is also described.
The document intends to capture and convey the significant architectural decisions,
which reflect into the current implementation or drive its development.\\
Some commonly used LArSoft elements are mentioned to exemplify
flows and connections, but this is no attempt to exhaustively describe
each, or any, of the single elements.

This document describes the architecture of LArSoft to date.
At the time of writing, LArSoft \texttt{v05\_00\_00} is in Release Candidate 2.
