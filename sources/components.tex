% \section{Logical view: components}
% \label{sec:Components}

%
% This is the merging of "component view" and "component interafctions"
% from J&M outline, following Saba's suggestion.
%
%

To provide the best solutions for LAr TPC simulation, reconstruction and
analysis of data, LArSoft provides a collection of built in tools and
algorithms, but also interfaces to other existing libraries.
\Cref{fig:LArSoftRelations} illustrates the relation between LArSoft and
these libraries.

\begin{figure}
	\includegraphics[width=\textwidth]{figures/LArSoftEnvironmentFull}
	\caption{\label{fig:LArSoftRelations}
		Relationship between LArSoft and other packages and libraries.
	}
\end{figure}

LArSoft is designed to rely on the \ART framework~\cite{ART}.
This framework provides an event data model, centralized configuration, data
and configuration persistency, management of user algorithm through
``modules'', exception handling, and more. LArSoft takes advantage of
the libraries \ART framework depends on, by using them directly:
\begin{itemize}
\item
  FHiCL language~\cite{FHiCL} to propagate the configuration to its
  components
\item
  message facility~\cite{MessageFacility} to regulate text output to console
\item
  CLHEP, ROOT, Boost libraries as needed
\end{itemize}

LArSoft also shares a common platform, \emph{nutools}~\cite{nutools}, with
other neutrino experiments (\NOvA). This library provides LArSoft with
some basic event display facilities and simulation data structures.

LArSoft interactions include:

\begin{itemize}
\item
  experiment detector data, through customized input modules converting
  data into LArSoft data classes
\item
  external event generators (e.g., CRY~\cite{CRY}, GENIE~\cite{GENIE}, via
  API or through HEPEVT exchange format; additional event generators are
  natively implemented in LArSoft
\item
  GEANT~\cite{GEANT}, an external detector simulation library
\item
  data bases, via direct connection or web proxy (\emph{libwda}~\cite{libwda})
\item
  external reconstruction tools (e.g., \emph{pandora}~\cite{pandora}), through a
  LArSoft interface
\item
  custom analysis tools, that can use LArSoft data classes directly or
  tailored data formats produced by custom LArSoft modules
\item
  experiment code, written in the form of LArSoft algorithms, modules
  and services
\end{itemize}

